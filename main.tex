\documentclass[10 pt]{beamer}

\usepackage{graphicx}

\usepackage{bm}
\usepackage{mathrsfs}
\usepackage{amsmath, amssymb}
\usepackage{color}
\usepackage{xcolor}
\usepackage[utf8]{inputenc}
\usepackage{setspace}
\usepackage{palatino}
\usepackage{graphicx}
\usepackage{float}
\usepackage{amsmath}
\usepackage{amssymb}
\usepackage{fancybox}
\usepackage{algorithm}
\usepackage{amsthm}
\usepackage{subcaption}
%\usepackage{algorithmic}
\usepackage{algpseudocode}
\usepackage{subcaption}
\usepackage{booktabs}
\fontfamily{ppl}\selectfont 

\setbeamertemplate{footline}[frame number]
\newcommand{\R}{{\mathbb R}}
\newcommand{\N}{{\mathbb N}}
\newcommand{\Q}{{\mathbb Q}}
\newcommand{\Z}{{\mathbb Z}}
\newcommand{\C}{{\mathbb C}}
\newcommand{\E}{{\mathbb E}}
\newcommand{\Cov}{\text{Cov }}
\newcommand{\Cor}{\text{Cor }}
\newcommand{\Var}{\text{Var }}

\mode<presentation>
{
  \usetheme{Berkeley}      % or try Darmstadt, Madrid, Warsaw, ...
  \usecolortheme{spruce} % or try albatross, beaver, crane, ...
  \usefonttheme{serif}  % or try serif, structurebold, ...
  \setbeamertemplate{navigation symbols}{}
  \setbeamertemplate{caption}[numbered]
} 

\title[]{Characterizing Identifiability of Models with Spatially Varying Unobserved Confounder}
\subtitle{\vspace{2mm} Updatesa
}


\author{Tommy Tang}

\institute[UIUC] % (optional)
{
  University of Illinois, Urbana Champaign
}


\date{}

\AtBeginSubsection[]
{
  \begin{frame}<beamer>{Outline}
    \tableofcontents[currentsection,currentsubsection]
  \end{frame}
}

\begin{document}

\frame{\titlepage}
\section{Introduction}

\begin{frame}{Current tasks}
Finishing proofs regarding $k$-distinguishability\bigskip

Classifying Matern covariance as $k$-distinguishable 
\bigskip

Extending results to linear model of coregionalization
\bigskip

Simulations
\end{frame}
\section{Lucky mistake}
\begin{frame}{Proof Error}
Subtle error claiming that $f(x_i) = g(x_i)$ at $k$ points $\{x_i\}$ implies $f'(x_i)=g'(x_i)$ at those same points \bigskip

Used in proofs of identification for exponential, Gaussian covariance functions
\end{frame}


\begin{frame}{General proposition for generalized polynomials}

Let $f:\R^+\to \R^+$ be any 1-1 function, and consider the family of functions
\[\mathcal{F} = \left\{\sum_{i=1}^k a_i f(x)^{d_i} : a_i, d_i \ge 0 \right\}.\]

Then $\mathcal{F}$ is scaled $2k$-distinguishable with respect to $(\R^+)^k$.\bigskip

Consider: the Gaussian or exponential covariance functions can be rewritten as:

\[\rho(w) = \sigma f(x)^{1/\phi}\] 

where $f(x) = e^{-w}$.

\end{frame}

\begin{frame}{$k$-distinguishable Families}
Call a family $\mathcal{F}$ of real-valued functions on $S$ $k$-distinguishable with respect to $R\subset S^{k}$, if $\forall (z_1,...,z_k) \in R$, and any pair of functions $f_1, f_2 \in \mathcal{F}$:
\[f_1(z_i) = f_2(z_i), \; \forall i\in\{1,...,k\} \iff f_1 = f_2 \text{ identically }.\]

Call a family $\mathcal{F}$ $k$-distinguishable with scaling with respect to some $R \subset S^k$ if  $\forall f_1, f_2 \in \mathcal{F}$, the existence of some $\alpha > 0$ and $(z_1,...,z_k)\in R$:

\[f_1(z_i) = \alpha f_2(z_i), \forall i\in \{1,...,k\}, \]

implies $f_1=f_2$ identically.

We are mostly interested in when $S^k \backslash R$ has null measure.

\end{frame}



\begin{frame}{Component specification }
The joint distributions of $U, Z$ is given by

\[\begin{pmatrix}U \\ Z\end{pmatrix} \sim \mathcal{N} \left(\begin{pmatrix}\mathbf{0} \\ \mathbf{0} \end{pmatrix}, \begin{pmatrix}\Sigma_U & \Sigma_{C} \\ \Sigma_{C} & \Sigma_{Z}\end{pmatrix} \right).\]

Under a separable covariance structure, we assume that

\[\Sigma_U = \sigma_U \Phi_U(\phi_U), \qquad\sigma_U \in \R, \phi_U \in \R^{k_U}, \Phi_U\in M_n(\R)\]
\[\Sigma_{C} = \sigma_{C} \Phi_{C}(\phi_{C}), \qquad\sigma_C \in \R, \phi_{C} \in \R^{k_{C}},\Phi_C\in M_n(\R)\]
\[\Sigma_Z = \sigma_Z \Phi_Z(\phi_Z), \qquad\sigma_Z \in \R, \phi_Z \in \R^{k_Z},\Phi_Z\in M_n(\R)\]
where the component covariance matrices have entries:

\[(\Phi_U)_{ij} = \rho_U(\phi_U,W_{ij}), \]
\[(\Phi_C)_{ij} = \rho_{C}(\phi_C, W_{ij}), \]
\[(\Phi_Z)_{ij} = \rho_{Z}(\phi_Z, W_{ij}), \]

and the distance matrix $W$ is considered known.

\end{frame}

\begin{frame}{Identification task}
Observe
\begin{align}
\Var Y|Z &= \Sigma_U - \Sigma_{C}\Sigma_Z^{-1} \Sigma_C + \epsilon^{-1} I_n \nonumber\\
\Var Z &= \Sigma_Z \nonumber\\
\E(YZ^T) \E(ZZ^T)^{-1} &= \beta_Z I_n + \Sigma_C \Sigma_Z^{-1} \nonumber \\
\end{align}
in order to identify the parameters:

\[\sigma_U, \sigma_Z, \sigma_C, \phi_u, \phi_z, \phi_c, \epsilon, \beta_Z.\]
\end{frame}




\begin{frame}{Special Cases (Gaussian and exponential)}

If $\rho_Z, \rho_C$ are either both Gaussian or both exponential, then the parameters $\beta_Z, \sigma_C, \phi_C$ are identifiable so long as there are four nonzero entries of $W$ and we assume $\Sigma_C = \Sigma_Z$. \\

\end{frame}

\begin{frame}{Special Cases (Linear sum families)}
Without restriction on $\rho_Z$, as long as $\Sigma_Z$ is identified and we assume $\Sigma_C = \Sigma_Z$, we may almost always identify covariance functions that parametrizable as linear sums.\bigskip

If $\rho_C$ can be written in the form \[\rho(\phi, w) = \sum_{i=1}^k \phi_i g_i(w)\] for any arbitrary preset $\{g_1,...,g_k\}$, then $\beta_Z, \sigma_C, \phi_C$ can be identified if there exist $k+1$ locations $(i_1,j_1),...,(i_{k+1},j_{k+1})$ with corresponding entries of $W$ and $\Sigma_Z$ (denoted $z_{ij}$) such that:

\[\begin{pmatrix}
z_{i_1,j_1} & g_1(w_{i_1,j_1}) & g_2(w_{i_1,j_1}) & \cdots & g_k(w_{i_1,j_1}) \\
z_{i_2,j_2} & g_1(w_{i_2,j_2}) & g_2(w_{i_2,j_2}) & \cdots & g_k(w_{i_2,j_2}) \\
\vdots & \vdots & \ddots & \vdots \\
z_{i_{k+1},j_{k+1}} & g_1(w_{i_{k+1},j_{k+1}}) & g_2(w_{i_{k+1},j_{k+1}}) & \cdots & g_k(w_{i_{k+1},j_{k+1}})
\end{pmatrix}\] is invertible. In particular we have identifiability if $\rho_C$ is spherical.
\end{frame}

\begin{frame}{Identification of $\sigma_\epsilon, \sigma_U, \phi_U$}

Assuming all other parameters are identified, we may identify the remaining parameters if $\rho_U$ is scaled $k$-distinguishable with respect to $R$ and $W$ contains at least $k$ distinct points of $R$.

\end{frame}

\section{Linear model of coregionalization}

\begin{frame}{Linear model of coregionalization}
Another method of guaranteeing positive definiteness. \bigskip

We model 

\[Z \sim a_1 S_1 + a_2S_2\]

\[U \sim b_1 S_1 + b_2S_2\]

where $S_1, S_2$ are independent spatial processes with covariances $\Sigma_1, \Sigma_2$ constructed out of covariance functions $\rho_1, \rho_2$. As before, we assume a fixed, known distance matrix $W$, with no multiplier. 
\end{frame}

\begin{frame}
We obtain the following expressions for observed values:
\begin{align*}
\Var Y|Z 
&= (b_1^2\Sigma_1+b_2^2\Sigma_2)+\epsilon^{-1} I_n \\ &+\left(\frac{1}{b_1^2}\Sigma_1^{-1} + \frac{a_2^2}{a^2_1b^2_1} \Sigma^{-1}_1\Sigma_2\Sigma^{-1}_1 \right)^{-1}\\ &+  \left(\frac{1}{b_2^2}\Sigma_2^{-1} +  \frac{a_1^2}{a^2_2b^2_2} \Sigma^{-1}_2\Sigma_1\Sigma^{-1}_2 \right)^{-1}\\ &+ 2\left( \frac{a_1}{a_2b_1b_2}\Sigma^{-1}_2 + \frac{a_2}{a_1b_1b_2}\Sigma^{-1}_1\right)^{-1}, \\
\Var Z &= a_1^2 \Sigma_1 + a_2^2 \Sigma_2, \\
\E(YZ^T)\E(ZZ^T)^{-1} 
&= (\beta_Z+b_1+b_2)I_n - \frac{b_1}{a_1}\left(\frac{a_1^2}{a_2^2} \Sigma_1\Sigma^{-1}_2 + I_n \right)^{-1} \\ &-\frac{b_2}{a_2}\left(\frac{a_2^2}{a_1^2} \Sigma_2\Sigma^{-1}_1 + I_n \right)^{-1}
.
\end{align*}
\end{frame}

\begin{frame}{Variance of treatment}
Is it possible to have distinct parameters that give the same variance?

\[a_1^2\Sigma_1 + a_2^2\Sigma_2 = \Var Z = a'^2_1\Sigma'_1 + a'^2_2\Sigma'_2\]

From the previous proposition, we may identify $a_1, a_2$ and the parameters to construct $\Sigma_1, \Sigma_2$ if the underlying covariance functions are exponential, Gaussian, or spherical and there are at least four distinct nonzero points on $W$.
\end{frame}

\begin{frame}{Further identification}
Let $M_1 = a_1^{-1}\left(\frac{a_1^2}{a^2_2} \Sigma_1\Sigma^{-1}_2 + I_n \right)^{-1}$, $M_2 = a_2^{-1}\left(\frac{a_2^2}{a^1_2} \Sigma_2\Sigma^{-1}_1 + I_n \right)^{-1}$, which are considered known after the previous identification.\bigskip
Assuming that $\Sigma_2\Sigma_1^{-1}$ is not diagonal, which implies $M_1, M_2$ not diagonal, then we may identify $b_1, b_2$ from observing the off-diagonal entries of $\E(YZ^T)\E(ZZ^T)^{-1}$ if there are two pairs of locations $(i,j), (l,k)$ such that:

\[\begin{pmatrix} 
(M_1)_{ij} & (M_2)_{ij}\\
(M_1)_{lk} & (M_2)_{lk}\
\end{pmatrix}\] is invertible. After $b_1, b_2$ are identified, we may then identify $\beta_Z$ from the diagonal entries.
\end{frame}

\section{Simulations}

\begin{frame}{Hierarchical Model for exponential, Gaussian, or spherical covariance}
\[\beta_Z \sim \mathcal{N}(0,\sigma_\beta^2)\]
\[\epsilon \sim \mathcal{N}(0,\sigma^2_\epsilon), \quad \sigma^2_\epsilon \sim \mathcal{N}(0, \alpha_\epsilon^2),\quad \alpha_\epsilon^2 \sim \mathcal{N}(0,1)\]
\[\sigma_U^2 \sim \mathcal{N}(0, \alpha_U^2), \quad \phi_U \sim \mathcal{N}(0, c_U^2), c_U^2 \sim \mathcal{N}(0,1)\]

\[\sigma_C^2 \sim \mathcal{N}(0, \alpha_C^2), \quad \phi_C \sim \mathcal{N}(0, c_C^2), c_C^2 \sim \mathcal{N}(0,1)\]

\[\sigma_Z^2 \sim \mathcal{N}(0, \alpha_Z^2), \quad \phi_Z \sim \mathcal{N}(0, c_Z^2), c_Z^2 \sim \mathcal{N}(0,1)\]
\[\Sigma_C = \sigma^2_C \cdot \rho_C(\phi_C,W)\]
\[Z \sim \mathcal{N}(0, \Sigma_Z); \Sigma_Z = \sigma^2_Z \cdot \rho_Z(\phi_Z,W)\]
\[U \sim \mathcal{N}(0, \Sigma_U); \Sigma_U = \sigma^2_U\cdot \rho_U(\phi_U,W)\] 

\[Y = Z\beta_Z + U + \epsilon\]

Should I be generating $U|Z$ since $Z$ and $Y$ are known?



\end{frame}

\section{Further}
\begin{frame}{Next steps}
Run Gibbs sampling 
\bigskip

Extend to Matern covariance function
\bigskip

Finish proof in discrete case
\bigskip

Consider real data
\end{frame}

\section{References}
\begin{frame}{References}
\tiny


Dupont, Emiko, Simon N. Wood, and Nicole H. Augustin. "Spatial+: a novel approach to spatial confounding." Biometrics (2022).
\bigskip

Dupont, Emiko, Simon N. Wood, and Nicole H. Augustin. "Rejoinder to the discussions of “Spatial+: A novel approach
to spatial confounding”" Biometrics (2022).
\bigskip

Hernán MA, Robins JM (2020). Causal Inference: What If. Boca Raton: Chapman \& Hall/CRC. \bigskip

Gilbert, Brian, Abhirup Datta, and Elizabeth Ogburn. "Approaches to spatial confounding in geostatistics." arXiv preprint arXiv:2112.14946 (2021).
\bigskip

Schnell, Patrick M., and Georgia Papadogeorgou. "Mitigating unobserved spatial confounding when estimating the effect of supermarket access on cardiovascular disease deaths." The Annals of Applied Statistics 14.4 (2020): 2069-2095.


\end{frame}

\end{document}

